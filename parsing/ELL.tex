\section{ELL Grammars}

A grammar is ELL iff:
\begin{itemize}
    \item It is ELR
    \item It has the Single Transition Property
    \item It has no Left Recursive Derivations (direct or indirect)
\end{itemize}

\subsection{Single Transition Property}
The STP does not hold if for an m-state there are more transitions possible for a given symbol.

\subsection{Parser Control Flow Graph}

From the machine net:
\begin{itemize}
    \item Add \emph{call arcs}: for $q_A \xrightarrow{B} r_A$ add $q_A \dashrightarrow 0_B$
    \item Add \emph{prospect sets} of the states
    \item Add the \emph{Gui} for the following arcs:
    \begin{itemize}
        \item Terminal shift arc: $q \xrightarrow{a} r$
        \item Call arcs: $q_A \dashrightarrow 0_B$
        \item Exiting arcs of the final states
        \item \textbf{Note}: \textbf{no} \emph{Gui} for nonterminal shift ($q_A \xrightarrow{B} r_A$)
    \end{itemize}
\end{itemize}

\subsection{Prospect sets}
The prospect set of state $q_X$ is the set of symbols that can be encountered when exiting the machine.
It is the union of the lookaheads of all the candidates with $q_X$ as kernel.

For an initial state $0_A$
\begin{equation*}
    \pi_{0_A} := \pi_{0_A} \cup \bigcup_{q_i \xrightarrow{A} r_i} (Ini(L(r_i)) \cup \ \text{if}\ Null(L(r_i)) \ \text{then}\ \pi_{q_i} \ \text{else}\ \emptyset)
\end{equation*}

For any other state $q$ (where $X_i$ is terminal or non-terminal)
\begin{equation*}
    \pi_q := \bigcup_{p_i \xrightarrow{X_i} q} \pi_{p_i}
\end{equation*}

\subsection{Guide set}
The guide set ($Gui$) of an arc is the set of symbols that can be met going through that arc.
\begin{itemize}
    \item $Gui(q_A\xrightarrow{a}r_A) = \{a\}$
    \item $Gui(f\rarr) =$ prospect set of $f$ (with $f$ final)
    \item $Gui(q_A\dashrightarrow 0_B) = Ini(L(B)) \cup \{\text{symbols following $B$ if $B$ is nullable}\}$
\end{itemize}

Formally
\begin{equation*}
    Gui(q_A \dashrightarrow 0_{A_1}) := \bigcup
    \begin{cases}
        Ini(L(A_1))\\ \\
        \text{if}\ Null(L(A_1)) \\
        \text{then}\ Ini(L(r_A)) \ \text{else}\ \emptyset\\ \\
        \text{if}\ Null(L(A_1)) \land Null(L(r_A)) \\
        \text{then}\ \pi_{r_A} \ \text{else}\ \emptyset \\ \\
        \bigcup_{0_{A_1} \dashrightarrow 0_{B_i}} Gui(0_{A_1} \dashrightarrow 0_{B_i})
    \end{cases}
\end{equation*}

\subsection{Alternative ELL Definition}
A grammar is ELL iff on alternative arcs the guide sets are disjoint.

\subsection{Top-Down Predictive Analyzer}
Based on the PCFG, using a stack initialized with $0_S$. If the top element is $q_A$ (machine $M_A$ is active at the state $q_A$), and the current character is $cc$:
\begin{description}
    \item[Scan] if $q_A \xrightarrow{cc} r_A$: read $cc$, \textbf{pop}, \textbf{push $r_A$}
    \item[Move] if $a_A \dashrightarrow 0_B$ and $q_A \xrightarrow{B} r_A$ and $cc \in Gui(q_A \dashrightarrow 0_B)$: \textbf{pop}, \textbf{push $r_A$}, \textbf{push $0_B$}
    \item[Return] if $q_A$ final, $cc$ in prospect set of $q_A$: \textbf{pop}
    \item[Acceptance] if $A=S$, $q_A$ is final and $cc = \dashv$: \textbf{accept}
\end{description}

\subsection{Increasing lookahead}
If a grammar is not ELL(1) it may be possible that it is ELL(k).
If the guide sets of length $k$ are disjoint the grammar is ELL(k).
